% タイトル
% 和文表題
\title[{\protect\LaTeX} による論文作成のガイド]%
	{{\protect\LATEX} による論文作成のガイド(第7.2版)}
% 英文表題
\etitle{How to Typeset Your Papers in {\LATEx} (Version 7.2)}
% 所属ラベルの定義
\affilabel{KU}{京都大学\\Kyoto University}
\paffilabel{Princeton}{プリンストン高等研究所(嘘です)\\
	Institute for Advanced Study, Princeton (just joke)}
\affilabel{NTT}{NTT基礎研究所\\NTT Basic Research Laboratories}
% 和文著者名
\author{中島 浩\affiref{KU}\affiref{Princeton}\and
	斉藤 康己\affiref{NTT} \and テスト}
	
% 英文著者名
\eauthor{Hiroshi Nakashima\affiref{TUT}\affiref{Princeton}\and
	Yasuki Saito\affiref{NTT}}

% 概要
% 和文概要
\begin{abstract}
テスト
このパンフレットは,情報処理学会論文誌(以後,論文誌と呼ぶ)に投稿する論文,
並びに掲載が決定した論文の最終版を,日本語 {\LaTeX} を用いて作成し提出するた
めのガイドである.このパンフレットでは,論文作成のためのスタイルファイルにつ
いて解説している.また,このパンフレット自体も論文と同じ方法で作成されている
ので,必要に応じてスタイルファイルとともに配布するソース・ファイルを参照され
たい.
\end{abstract}
% 英文概要
\begin{eabstract}
This pamphlet is a guide to produce a draft to be submitted to IPSJ Journal
and Transactions and the final camera-ready manuscript of a paper to appear
in the Journal\slash Transactions, using Japanese {\LaTeX} and special style
files.  Since the pamphlet itself is produced with the style files, it will
help you to refer its source file which is distributed with the style files.
\end{eabstract}

% 表題などの出力
\maketitle