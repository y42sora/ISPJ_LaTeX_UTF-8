% 1章 はじめに
\section{はじめに}
\footnotetext[1]{実際の著者は情報処理学会 論文誌編集委員会である.}
テスト
情報処理学会では,論文誌を迅速かつ低コストで出版するために {\LaTeX} による製版
を採用している.この製版方式では,著者が作成した {\LaTeX} ソースが基本的にはそ
のまま最終的な製版プロセスに使用される.したがって,多数の読者に親しまれてき
た体裁を継承し,読み易い論文誌を出版するためには,著者の方々の協力が不可欠で
ある.

一方,著者にとってのメリットとしては,活字製版では避け難い製版時の誤りがなく
なり,校正の手間が大幅に削減されることがあげられる.また専用のスタイルファイ
ルと通常使われる {\LaTeX} のコマンドを使えば,簡単に論文誌の体裁に則った出力
が得られるので,日頃 {\LaTeX} で文書を作成している多くの著者には無理なく受け
入れられるものと期待している.さらに,投稿用のスタイルファイルも用意されてお
り,最終版作成のための修正は最小限となるだけでなく,以前に比べて格段に読み易
い草稿を得ることができる.これは自分の原稿をチェックする著者だけではなく,査
読者にとっても大きなメリットである.

なお,論文誌スタイルには通常の {\LaTeX} に追加されたコマンドがあり,その多く
は論文製版に不可欠なものである.またスタイルファイルだけでは対処しきれない体
裁上の注意事項もいくつかある.したがって,著者も含めて論文誌作成に関わる全て
の人々の労力を軽減するためにも,原稿を作成する前にこのガイドを{\bf 良く読ん
で規定を厳密に守っていただきたい}.

%}{

