% 8
\section{研究会論文用コマンド}
\label{sig}

各研究会論文誌(トランザクション)には各々に固有のサブタイトル,略称,通
番がある.最終原稿では,以下のコマンドを \|\documentclass| の{\bf オプショ
ン}とすることで,これらの情報を与える.

\begin{itemize}
\item \|PRO|(プログラミング)
\item \|TOM|(数理モデル化と応用)
\item \|TOD|(データベース)
\item \|ACS|(コンピューティングシステム)
\item \|CDS|(コンシューマ・デバイス\,\&\,システム)
\item \|TBIO|(Bioinformatics)\footnote{%
TBIO, SLDM, CVAは英文論文誌であるので和名はない.}
\item \|SLDM|(System LSI Design Methodology)\footnotemark[5]
\item \|CVA|(Computer Vision and Applicaitons)\footnotemark[5]
\end{itemize}

また英文論文作成の際には \|english| をオプションに追加すればよい.したがっ
て,\|\documentclass[PRO]{ipsj}| とすれば「プログラミング」の和文用,
\|\documentclass[PRO,english]| \|{ipsj}| とすれば英文用となる.

また研究会には「号」と連動しない「発行月」があるため,学会あるいは編集委
員会の指示に基づき,発行月を
%
\begin{itemize}\item[]
\|\setcounter{|{\bf 月数}\|}{<発行月>}|
\end{itemize}
%
によって指定する.

この他,以下の各節で示すように,いくつかの論文誌に固有の機能を実現するた
めのコマンドなどが用意されている.

\newpage%%