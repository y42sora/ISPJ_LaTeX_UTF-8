% 2章 投稿から出版まで
\section{投稿から出版まで}\label{sec:Enum}\label{sec:item}
テスト
投稿する論文の作成から,論文が掲載された論文誌が出版されるまでの流れは,以下
の通りである\footnote[2]{%
%
ジャーナルの手順に沿った説明となっており,トランザクションでは個々に固有の異
なる手順が定められていることがある.詳細については各トランザクション編集委員
会に照会されたい.}
.
%
\begin{Enumerate}%{
\item {\bf スタイルファイルの取得}\\
情報処理学会のWEB site \|http://www.ipsj.or.jp/|から,スタイルファイルなど
からなる論文作成キットをダウンロードすることができる.このキットには以下のファ
イルが含まれている\footnote[3]{%
%
下記のほかに非日本語環境でのBib{\TeX}スタイルとして,\texttt{ipsjsort-e.bst}
と\texttt{ipsjunsrt-e.bst}も含まれている.}.
%
\begin{itemize}%{
\item\|ipsjpapers.sty| : 製版用スタイル
\item\|ipsjpapers.cls| : {\LATEXe} 用製版用スタイル
\item\|ipsjdrafts.sty| : 投稿用スタイル
\item\|ipsjcommon.sty| : 製版/投稿用補助スタイル
\item\|ipsjsort.bst  | : jBib{\TeX}スタイル(著者名順)
\item\|ipsjunsrt.bst | : jBib{\TeX}スタイル(出現順)
\item\|sample.tex    | : このガイドのソース(製版用)
\item\|dsample.tex   | : このガイドのソース(投稿用)
\item\|esample.tex   | : 英文ガイドのソース(製版用)
\item\|desample.tex  | : 英文ガイドのソース(投稿用)
\item\|bibsample.bib | : 文献リストのサンプル
\item\|ebibsample.bib| : 英文文献リストのサンプル
\end{itemize}%}
%
キットは Unix 用,Windows (DOS) 用,Macintosh 用などが用意されており,著者の
作業環境に応じたものを選択できるようになっている.

\item {\bf 投稿用原稿の作成と投稿}\\
このガイドにしたがって,後述の \|draft| オプションを指定した {\LaTeX} ソース
を作成し,その \|.dvi| ファイルをPDFファイルに変換する.
なお著者の氏名・所属,著者紹介,謝辞は投稿用原稿に含まれていてはならないが,
後述するコマンド等を用いて指定していれば自動的に出力が抑止される.
PDFファイルを投稿するにはまず,
\begin{itemize}\item[]\tt
http://www.ipsj.or.jp/08editt/journal/submit/
\end{itemize}
にアクセスして投稿情報を登録し,その結果送られてくるemailに記載のURLをアクセ
スする.

\item {\bf 製版用原稿の作成}\\
採録が決定したら,査読者からのコメントなどにしたがって原稿を修正し,著者紹介
など投稿時になかった項目があれば追加する.また図表などのレイアウトも最終的な
ものとする.なお後の校正の手間を最小にするために,{\bf この段階で記述の誤り
などを完全に除去するように綿密なチェックをお願いしたい}.

\item {\bf 製版用原稿とファイルの送付}\\
学会へは {\bf {\LaTeX} ファイル(をまとめたもの)とハードコピーの双方を}送付
する.送付するファイル群の標準的な構成は \|.tex| と \|.bbl| であり,この他に 
PostScript ファイルや特別なスタイルファイルがあれば付加する.なお \|.tex| は
印刷業者が修正することがあるので,{\bf 必ず一つのファイルにしていただきたい}.
また必要なファイルが全てそろっていること,特に特別なスタイルファイルに洩れが
ないことを,注意深く確認して頂きたい.

ファイルの送付方法などについては,採録通知とともに学会事務局から送られる指示
にしたがっていただきたい.

\item {\bf 著者校正}\\
学会では用語や用字を一定の基準にしたがって修正することがあり,また {\LaTeX}
の実行環境の差異などによって著者が作成したハードコピーと実際の製版結果が微妙
に異なることがある.これらの修正や差異が問題ないかを最終的に確認するために,
著者にゲラ刷りが送られるので,もし問題があれば朱書によって指摘して返送する.
なお{\bf この段階での記述誤りの修正は原則として認められない}ので,原稿送付時
に細心の注意を払っていただきたい.

\item {\bf 製版・出版}\\
著者の校正に基づき最終的な製版を行ない,オンライン出版する.
\end{Enumerate}%}

%}{

