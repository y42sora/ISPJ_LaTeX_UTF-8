% 付録1 研究会論文誌用コマンド
\section{研究会論文誌用コマンド}\label{sec:app-trans}
テスト
各研究会論文誌(トランザクション)には各々に固有のサブタイトル,略称,通番が
ある.製版用原稿では,以下のコマンドを\|\begin{document}|の前に置くことによ
り,これらの情報を与える.
%
\begin{itemize}\item[]
\|\transaction{|\<略称\>\|}{|\<巻数\>\|}{|\<号数\>\|}|
\end{itemize}
%
なお\<略称\>は以下のいずれかであり,\<巻数\>と\<号数\>は学会あるいは研究会論
文誌編集委員会の指示に基づいて与える.
%
\begin{itemize}%{
\item
\|PRO|(プログラミング)
\item
\|TOM|(数理モデル化と応用)
\item
\|TOD|(データベース)
\item
\|ACS|(コンピューティングシステム)
\item
\|CVIM|(コンピュータビジョンとイメージメディア)
\item
\|TBIO|(Bioinformatics)\footnote{%
TBIO, SLDM, CVAは英文論文誌であるので和名はない.}
\item
\|SLDM|(System LSI Design Methodology)\footnotemark[1]
\item
\|CVA|(Computer Vision and Applicaitons)\footnotemark[1]
%
\end{itemize}%}
%
また上記の\<号数\>は発行月とは連動していないので,学会あるいは編集委員会の指
示に基づき発行月を;
%
\begin{itemize}\item[]
\|\setcounter{|{\bf 月数}\|}{|\<発行月\>\|}|
\end{itemize}
%
によって指定する.

この他,以下の各節で示すように,いくつかの論文誌に固有の機能を実現するための
コマンドなどが用意されている.

%}{

\subsection{「プログラミング」固有機能}

「論文誌:プログラミング」には論文以外に,プログラミング研究会での研究発表の
内容梗概が含まれている.この内容梗概は,\|\documentclass|または
\|\documentstyle|のオプションとして\|abstract|を指定した上で,
\ref{sec:config}節の\|\maketitle|までの内容からなるファイル(すなわち本文が
ないファイル)から生成する.なお\|\|{\bf 受付}や\|\|{\bf 採録}は不要であるが,
代わりに発表年月日を;
%
\begin{itemize}\item[]
\|\|{\bf 発表}\|{|\<年\>\|}{|\<月\>\|}{|\<日\>\|}|
\end{itemize}
%
により指定する.

%}{

\subsection{「数理モデル化と応用」固有機能}

「論文誌:数理モデル化と応用」の論文では,受付や採録の日付以外に再受付日付を
記載するように指示されることがある.その場合には;
%
\begin{itemize}\item[]
\|\|{\bf 再受付}\|{|\<年\>\|}{|\<月\>\|}{|\<日\>\|}|
\end{itemize}
%
により指定する.なお複数回の再受付が行われた場合,上記のコマンドを繰り返し使
用する.

%}{

\subsection{「データベース」固有機能}

「論文誌:データベース」の論文の担当編集委員の氏名は;
%
\begin{itemize}\item[]
\|\edInCharge{|\<氏名\>\|}|
\end{itemize}
%
により指定する.

%}{

\subsection{「Bioinformatics」固有機能}

Trans.\ Bioinformatics (TBIO)に固有の機能を利用するためには,\|\documentclass|
(または\|\documentstyle|)のオプションで\|TBIO|を指定する.なおTBIOは英文論
文誌であるので,\|TBIO|オプションの指定によって自動的に\|english|オプション
が指定されたものとみなされる.したがって;
%
\begin{itemize}\item[]
\|\documentclass[TBIO]{ipsjpaper}|
\end{itemize}
%
のように\|english|オプションを省略することができる.またこのオプションの指定
により,以下のコマンドが使用可能となる.
%
\begin{itemize}%{
\item
論文の種別は;
%
\begin{itemize}\item[]
\|\TBIOpapercategory{|\<種別\>\|}|
\end{itemize}
%
を用いて,\|original|, \|survey|, \|database| のいずれかを指定する.この結果,
``{\it Original Paper}'', ``{\it Survey Paper}'' または ``{\it
Database\slash Software Paper}'' のいずれかが先頭ページのタイトルの左上に表
示される.なおこのコマンドが与えられなければ \|original| とみなされる.

\item
担当編集委員の氏名は;
%
\begin{itemize}\item[]
\|\edInCharge{|\<氏名\>\|}|
\end{itemize}
%
により指定する.

\item
査読過程で条件付採録となった論文の再受付日は;
%
\begin{itemize}\item[]
\|\rereceived{|\<年\>\|}{|\<月\>\|}{|\<日\>\|}|
\end{itemize}
%
により指定する.

\end{itemize}%}
%
なお後の2つのコマンドの使用はオプショナルであり,学会あるいは編集委員会から
の情報提供や指示がなければ省略しても構わない.

%}{
